% Chapter Template

\chapter{Applications For Monomial Deciders} % Main chapter title

\label{ChapterX} % Change X to a consecutive number; for referencing this chapter elsewhere, use \ref{ChapterX}


%----------------------------------------------------------------------------------------
%	SECTION 2
%----------------------------------------------------------------------------------------

\section{Starting With A Theorem Of Infiniteness}

This section assumes that both P=NP and P!=NP and the following sections will provide reasoning and examples.

\section{Mapping Out Representations}

This section contains my opinions of what mathematics is.

\section{Euler's Constant}

Euler's constant is an example of P = NP because of it's use of calculus.

$\\ $

To show that it is also in the problem set of P $\neq$ NP, start by using the picking function going into infinity.

$\\ $

$e = pf(x) = \theta(language\ of\ pf(x))$

$\\ $

$\left\{ x = x^2/x = x^3/x^2 = x^4/x^3 = x^5/x^4 = \cdots = x^n/x^{n-1} \right\}$

$\\ $

$1 + 1 + 1/(1+1) + 1/(3+3) + 1/(4+(4*3+4*2)+4)$

\section{Sketching Into Code}

There is a technique to develop code from a diagram of a decider. Take into account the degree of the states and that should account for the halting required to break from the loop.

\begin{lstlisting}
bool generalizedMD(int y)
{
    if (y == 0)
    {
        return true;
    }
    var s = y;
    while (s >= 0)
    {
        for (int i = 0; i < 2; i++)
        {
            for (int j = 0; j < 4; j++)
            {
                if (s == 0 && (j == 0 || j == 2) && i == 0)
                {
                    return true;
                }
                else if (s < 0)
                {
                    return false;
                }
                s -= 1;
            }
        }
    }
    return true;
}
\end{lstlisting}

\section{Gather Some Data}

Using the generalizedMD function where we make some sketch of the initial algorithm, we can collect some data Generate: 

f(x) = y on the first line

Negatives on the second

$\\ $

We notice that we need two finishing states and that all the ven y's end in one state and all the odd y's end on the other.

E: Is it even?

$\\ $

We create a for loop of $n^3$ that with constant 2 and count how many times it passes through the finishing state.

T: Total amount of times it passes the finishing state.

$\\ $

Insight is gained by noticing that the difference only increases every other time and that it increases by a difference of 2 every it passes a finishing state.

D: Difference of the number of hits to the finishing state between the alst time it is called and the first.

$\\ $

\begin{lstlisting}
// Generates negatives from a general monomial decider represented as an algorithm so
// that we can collect data about the negatives
int[] Generator(int max)
{
    int[] result = new int[max + 1];
    int x = 0;
    int negatives = 0;
    int i = 0;
    while (x < max + 1)
    {
        int num = 2 * (Convert.ToInt32(Math.Pow(x, 2)));
        if (generalizedMD(i))
        {
            // A simple verifier
            if (num == i)
            {
                Console.WriteLine(negatives);
                Console.WriteLine("f(" + x + ") = " + i);
                result[x] = i;

                x++;
                negatives = 0;
            }
            else
            {
                negatives++;
            }
        }
        i++;
    }

    return result;
}
\end{lstlisting}

\section{Representing Monomial Deciders As Code}

$f(x)\ =\ 2x^2 = \left\{ 0,2,8,18,\cdots \right\}$

$\\ $

x is even at 0,8,32,72

x is odd at 2,18,50

$\\ $

There are four variables constructed:

Current hits records the number of times path traveled hits a finishing state.

Total number of times traveled on a finishing state needed to reach a valid decision.

Diff is the current number of diff to increment total hits by.

IsEven is if this resets back to even, increment Diff by two.

$\\ $

The following is code generated from our more formal representation of the solution.

\begin{lstlisting}
// After getting some log results, we can construct a decider
bool MonomialDecider2xx(int y)
{
    var total = 0;
    var hits = 0;
    var diff = 1;
    var isEven = 0;
    var s = 0;

    while (s <= y)
    {
        for (int i = 0; i < 2; i++)
        {
            for (int j = 0; j < 2; j++)
            {
                for (int k = 0; k < 2; k++)
                {
                    if ((i == 0)&&(j == 0 || j == 1)&&(k == 0))
                    {
                        if (hits == total)
                        {
                            if (s == y)
                            {
                                Console.WriteLine(s + ": Hits: " + total);
                                return true;
                            }
                            else if (s > y)
                            {
                                return false;
                            }

                            total += diff;
                            isEven++;
                            if (isEven % 3 == 2)
                            {
                                isEven = 0;
                                diff += 2;
                            }
                        }

                        hits++;
                    }
                    s++;
                }
            }
        }
    }

    return false;
}
\end{lstlisting}

\section{Negative Numbers}

Representing negative numbers can be thought of discretely. Below is a representaation of negative numbers.

$\\ $

Cancellation

$\\ $

Addition

$\\ $

Start and Finish: $2x$

$\\ $

Start and Finish: $x^2$

$\\ $

Ignoring the rules for commutativity and association for now.

\section{Pi}

Representing the constant pi, $\pi$, in the language of polynomials using the Leibniz formula $\pi/4 = 1 - 1/2 + 1/5 -1/7 + 1/9 + \cdots$

\section{Fibonacci}

Given the fibonacci sequence, ${1,1,2,3,5,8,13,21,\cdots}$, representing this sequence as a monomial generator in the language of polynomials can be shown below.

$\\ $

From the circles above, we see that f(1) = 1 and f(2) = 2. If we add f(1) and f(2) together we get f(3) = 3 and so on and so forth.

$\\ $

Notice that on odd inputs, Ex, there aren't any circles in the circles/states but in even inputs on X, there are two states. On outputs, Ey, it is odd twice then even once.

\begin{lstlisting}
int fibonacci(int n)
{
    if (n == 0)
    {
        return 0;
    }

    int y = 1;
    int y1 = 1;
    int y2 = 0;

    for(int i = 1; i < n; i++)
    {
        y = y1 + y2;
        y2 = y1;
        y1 = y;
    }

    return y;
}
\end{lstlisting}

\section{Analysis Of Parity In Fibonacci}

On analyzing the parity of Ex and Ey of the fibonacci sequence, we that there are two patterns.

$\\ $

$\begin{array}{ccc}
E_1 & E_2 & E_3\\
0 & 1 & 0\\
1 & 0 & 1\\
0 & 1 & 1\\
\end{array}$

$\\ $

$Det(E_1,E_2,E_3) = -1$ 

$Det(E_2,E_1,E_3) = 1$

$\\ $

$-1 + 1 -1 = -1$

$1 - 1 + 1 = 1$

\section{Redrawing the Fibonacci Sequence}

From our analysis, it can be seen that there are three states that are a minimum to create a fibonacci sequence. Minimization gives us a monomial generator, ${S_x,S_y,S_z}$. Set the three states to a desired configuration and it will generate the fibonacci sequence. It can shown that it requires three states minimum to generate the fibonacci sequence.

$\\ $

Generator

$S_x = S_y + S_z$

$S_y = S_z + S_x$

$S_z = S_x + S_y$

Using the above, dynamic programming can be modeled as a set of generator functions.

\begin{lstlisting}
int fibonacciGenerator(int n)
{
    int stateX = 0;
    int stateY = 1;
    int stateZ = 1;
    int cycles = 0;

    while (cycles <= n)
    {
        cycles++;

        if (cycles > n)
        {
            return stateX;
        }

        stateX = stateY + stateZ;
        Console.WriteLine("stateX: " + stateX + "\tstateY: " + stateY + "\tstateZ: " + stateY);
        
        cycles++;

        if (cycles > n)
        {
            return stateY;
        }

        stateY = stateX + stateZ;
        Console.WriteLine("stateX: " + stateX + "\tstateY: " + stateY + "\tstateZ: " + stateY);

        cycles++;

        if (cycles > n)
        {
            return stateZ;
        }

        stateZ = stateX + stateY;
        Console.WriteLine("stateX: " + stateX + "\tstateY: " + stateY + "\tstateZ: " + stateY);
    }

    return stateX;
}
\end{lstlisting}

\section{The Fibonacci Decider}

Given the two monomial deciders we found using the determinant, we know that we must have six numbers in the sequence to decide if they form a Fibonacci sequence. In order to create this decider, we use an addition operator to merge them together because they cancel each other out. All deciders in each set must be true in order for the sequence to be a valid Fibonacci sequence.

\section{The Fibonacci Picking Function}

Now, let's apply the picking function, pf, to the fibonacci decider and show the probability, $P_r$, of finding a sequence. $pf({xyz}_1 + {xyz}_2) = Pr({xyz}_1)*Pr({xyz}_2) \leq 1/n^{k}$

Each decider has 3 representations giving $3^2 = 9$ total for each set. We pick one decider in each set to get the probability.

There are 6 deciders with 6 permutations giving $6^2 = 36$ permutations.

This shows a concrete example of the picking function which is a type of one way function.
